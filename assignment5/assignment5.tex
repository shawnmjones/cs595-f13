\documentclass[letterpaper,11pt]{article}
\usepackage{graphicx}
\usepackage{listings}
\usepackage[super]{nth}
\usepackage[hyphens]{url}
\usepackage{amsmath}
\usepackage[makeroom]{cancel}
\usepackage[table]{xcolor}
\usepackage{comment}
\usepackage[space]{grffile}

\lstset{
	basicstyle=\footnotesize,
	breaklines=true,
}

\begin{document}

\begin{titlepage}

\begin{center}

\Huge{Assignment 5}

\Large{CS 595:  Introduction to Web Science}

\Large{Fall 2013}

\Large{Shawn M. Jones}

\Large Finished on \today

\end{center}

\end{titlepage}

\newpage
\section*{1}

\subsection*{Question}

\begin{verbatim}
1.  Determine if the friendship paradox holds for your Facebook account.  
Create a graph of the number of friends (y-axis) and the friends sorted
by number of friends (x-axis).  (The friends don't need to be labeled 
on the x-axis.)  Do include yourself in the graph and label yourself
accordingly.

Compute the mean, standard deviation, and median of the number of friends
that your friends have.

You can download your network in an XML file by using the NameGenWeb
Facebook app: 

https://apps.facebook.com/namegenweb/ 

You will need to give this app permission to access your Facebook
data. Make sure you select "Friend Count" as an Extended Attribute.
When you download the data, download it in the GraphML format.

If you do not have a Facebook account, email me and I will send you 
my GraphML file.
\end{verbatim}

\newpage
\subsection*{Answer}


\newpage
\section*{2}

\subsection*{Question}

\begin{verbatim}
2.  Determine if the friendship paradox holds for your Twitter account.
Since Twitter is a directed graph, use "followers" as value you measure
(i.e., "do your followers have more followers than you?").

Generate the same graph as in question #1, and calcuate the same 
mean, standard deviation, and median values.

For the Twitter 1.1 API to help gather this data, see:

https://dev.twitter.com/docs/api/1.1/get/followers/list

If you do not have followers on Twitter (or don't have more than 20),
then use my twitter account "phonedude_mln".
\end{verbatim}

\newpage
\subsection*{Answer}

\newpage
\section*{3}

\subsection*{Question}

\begin{verbatim}
Extra credit, 2 points:

3.  Repeat question #1, but with your LinkedIn profile.
\end{verbatim}

\newpage
\subsection*{Answer}

\newpage
\section*{4}

\subsection*{Question}

\begin{verbatim}
Extra credit, 1 point:

4.  Repeat question #2, but change "followers" to "following"?  In
other words, are the people I am following following more people?
\end{verbatim}

\newpage
\subsection*{Answer}


\clearpage
\bibliographystyle{acm}
\bibliography{references}

\end{document}